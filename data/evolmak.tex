\begin{quote}
Tabla 1.3. Etapas propuestas en la evolución conceptual del marketing
\end{quote}

\begin{longtable}[]{@{}llllll@{}}
\toprule
\begin{minipage}[b]{0.14\columnwidth}\raggedright
\strut
\end{minipage} & \begin{minipage}[b]{0.14\columnwidth}\raggedright
Bartels

(1988)\strut
\end{minipage} & \begin{minipage}[b]{0.14\columnwidth}\raggedright
Munuera

(1992)\strut
\end{minipage} & \begin{minipage}[b]{0.14\columnwidth}\raggedright
Vargo y Lusch (2004)\strut
\end{minipage} & \begin{minipage}[b]{0.14\columnwidth}\raggedright
Kumar

(2015)\strut
\end{minipage} & \begin{minipage}[b]{0.14\columnwidth}\raggedright
El Ansary

(2017)\strut
\end{minipage}\tabularnewline
\midrule
\endhead
\textbf{Hasta 1900} & Antecedentes & Período de identificación & Teoría
económica clásica & No considerada & No considerada\tabularnewline
\textbf{1900-10} & Descubrimiento & & & &\tabularnewline
\textbf{1910-20} & Conceptual. & & & & Época del paradigma
tradicional\tabularnewline
\textbf{1920-30} & Integración & Periodo Funcionalista & &
&\tabularnewline
\begin{minipage}[t]{0.14\columnwidth}\raggedright
\strut
\end{minipage} & \begin{minipage}[t]{0.14\columnwidth}\raggedright
\strut
\end{minipage} & \begin{minipage}[t]{0.14\columnwidth}\raggedright
\strut
\end{minipage} & \begin{minipage}[t]{0.14\columnwidth}\raggedright
Periodo de

formación del

marketing\strut
\end{minipage} & \begin{minipage}[t]{0.14\columnwidth}\raggedright
\strut
\end{minipage} & \begin{minipage}[t]{0.14\columnwidth}\raggedright
\strut
\end{minipage}\tabularnewline
\textbf{1930-40} & Desarrollo & & & Marketing como economía aplicada
&\tabularnewline
\textbf{1940-50} & Nueva Estimación & & & Marketing como actividad
directiva &\tabularnewline
& & Periodo Conceptual & & & Paradigma macro a micro\tabularnewline
\textbf{1950-60} & Reformulación & & & Marketing como ciencia
cuantitativa &\tabularnewline
\begin{minipage}[t]{0.14\columnwidth}\raggedright
\strut
\end{minipage} & \begin{minipage}[t]{0.14\columnwidth}\raggedright
\strut
\end{minipage} & \begin{minipage}[t]{0.14\columnwidth}\raggedright
\strut
\end{minipage} & \begin{minipage}[t]{0.14\columnwidth}\raggedright
Gestión

de marketing\strut
\end{minipage} & \begin{minipage}[t]{0.14\columnwidth}\raggedright
\strut
\end{minipage} & \begin{minipage}[t]{0.14\columnwidth}\raggedright
\strut
\end{minipage}\tabularnewline
\textbf{1960-70} & Diferenciación & Periodo de definiciones formales & &
Marketing como ciencia conductual & Expansión genérica del
paradigma\tabularnewline
\textbf{1970-80} & Socialización & & & Marketing como ciencia de la toma
de decisiones &\tabularnewline
\begin{minipage}[t]{0.14\columnwidth}\raggedright
\textbf{1980-90}\strut
\end{minipage} & \begin{minipage}[t]{0.14\columnwidth}\raggedright
Conceptual. Actual\strut
\end{minipage} & \begin{minipage}[t]{0.14\columnwidth}\raggedright
\strut
\end{minipage} & \begin{minipage}[t]{0.14\columnwidth}\raggedright
Marketing como

proceso

Social

y económico\strut
\end{minipage} & \begin{minipage}[t]{0.14\columnwidth}\raggedright
Marketing como ciencia integrada\strut
\end{minipage} & \begin{minipage}[t]{0.14\columnwidth}\raggedright
\strut
\end{minipage}\tabularnewline
\textbf{1990-00} & & & & Marketing como recurso escaso &\tabularnewline
\textbf{05-12} & & & & Marketing como inversión &\tabularnewline
\textbf{12-} & & & & Marketing como parte central de la organización
&\tabularnewline
\bottomrule
\end{longtable}

Fuente: Bartels (1988), Munuera (1992), Kumar (2015), Vargo y Lusch
(2004)
